\documentclass[11pt]{article}
\usepackage[top=1.00in, bottom=1.0in, left=1.1in, right=1.1in]{geometry}
% \usepackage{Sweave}
\renewcommand{\baselinestretch}{1.1}
\usepackage{graphicx}
\usepackage{natbib}
\usepackage{amsmath}
\usepackage{gensymb}
\usepackage{parskip}
\usepackage{xcolor}
\usepackage[disable]{todonotes}
\usepackage{xr-hyper}
\usepackage[utf8]{inputenc}
\externaldocument{workingdraft}

\def\labelitemi{--}
\parindent=0pt

\begin{document}
% \SweaveOpts{concordance=FALSE}
\renewcommand{\refname}{\CHead{}}

% See also: git/grants/crc2023/crc2023app/docs/crc_notes2023more.tex which has some reference notes

\title{Supplement: Why longer seasons with climate change may\\ not increase tree growth}
\author{Team Grephon}
\date{\today}
\maketitle

\renewcommand{\thetable}{S\arabic{table}}
\renewcommand{\thefigure}{S\arabic{figure}}

<<echo=FALSE>>=
source("~/Documents/git/projects/grephon/grephon/analyses/whathappened.R")
@

\section*{Literature review methods}

We conducted a review to find studies focused on relationships between growing season length and tree wood growth, though contrasting terminology made it challenging to identify papers through one search. After reviewing several recent papers \citep{dow2022warm,zohner2023effect}, we searched ISI Web of Science for ``growing season length" AND ``tree ring*" (ALL FIELDS) on 12 April 2023, which returned 33 citations. We next reviewed abstracts and discarded papers that did not mention the relationship between growing season length and growth. We further reviewed all citations within all papers for additionally relevant papers and included them in our review. In total we report on \Sexpr{papernum} papers after reviewing over 107 potentially relevant papers, after discarding one paper \citep[][which used tree lines as a metric of both growth and growing season length]{bruening2017}. % These ISI methods in outline.tex % grephon_warmingexpts also has some ISI searches, but no date and is focused on warming experiments. 107 is the number in the articles folder. 
% on 31 Oct 2023, I tried ISI (Web of Science Core Collection) for SEARCH tree OR woody (All Fields) and growth (All Fields) and growing season length (All Fields), which returned 886 references. 
% Average number of papers in ecological meta-analysis: 24 papers see https://esajournals.onlinelibrary.wiley.com/doi/abs/10.1002/ecy.3680

Given the large diversity of metrics we found, we did not extract quantitative estimates of growing season length, growth, or their relationship. Instead, we extracted data on location, species, how they measured growing season length, growth, what relationship they found and what internal and external drivers they mentioned (full dataset with more details available on the Knowledge Network for Biocomplexity at publication).

Papers often reported dozens or more statistical tests that were variations of their data, thus we recorded a unique meta-analytic observation (i.e., one row within of data) within each paper (a `study' ID) when papers reported: (1) distinctly different datasets (e.g., a global analyses of observations and a short-term experiment); (2) multiple distinctly different measures of growth (e.g., tree ring width and flux tower) and/or growing season length (e.g., they reported both end of season as budset and end of wood growth through xylogenesis); (3) distinctly different results for growth $\times$  growing season length depending on metric (e.g., using budset for growing season length they find a growth $\times$ growing season length, but using leaf coloring they do not). 

% Methods below adapted from https://github.com/lizzieinvancouver/grephon/issues/29
We also assessed all papers for which hypotheses they addressed. For this, we reviewed papers again for stated hypotheses and/or analogous research questions that were specifically addressed in the results (e.g. hypothesis testing).  We found one paper that did not provide a clear hypothesis. After this we grouped each hypothesis to a broader main hypothesis (or hypothesis cluster) displayed in Fig. \ref{fig:hypotheses}. We extracted all hypotheses/questions from each paper, resulting in many papers had more than one hypothesis (\Sexpr{nrow(morethanonehyp)} of \Sexpr{papernum} papers). % ADDPAPER with no hypothesis

\subsection*{Trends with year}
Across studies which found support for or against a relationship between growth and growing season length (\Sexpr{nrow(yesandnoeviany)} of \Sexpr{nrow(d)} total studies) the range of years was very similar: spanning \Sexpr{quantile(eviany$year)[1]} to \Sexpr{quantile(eviany$year)[5]} with a mean of \Sexpr{round(mean(eviany$year),0)} for studies that found a positive relationship and spanning \Sexpr{quantile(noeviany$year)[1]} to \Sexpr{quantile(noeviany$year)[5]} with a mean of \Sexpr{round(mean(noeviany$year),0)} for studies that did not. Further, we found no trend through time for finding a positive or negative relationship through logistic regression (slope strongly overlapped 0). 

\section{Growth $\times$ elevation relationships}

Will add METHODS here ....

\section{The challenge of metrics: Measuring growth}

Tree growth, which can be measured in a variety of ways, highlights how the diversity of current metrics slows a unified model of when growth increases with longer seasons. Our literature review found that most studies quantified growth by measuring radial growth (e.g., through increment cores or dendrometers, $n=$\Sexpr{boxradg}), but a number also looked at metrics related to C assimilation (e.g. net ecosystem productivity or gross primary productivity, $n=$\Sexpr{boxcaccg}), while a smaller number examined biomass, height, or number of stems ($n=$\Sexpr{boxbiomassg}), or root:shoot ratio ($n=$\Sexpr{boxrootshootg}). Some studies used modeled estimates of photosynthesis 
(e.g., \citet{smith2014implications} relied on daily photosynthesis estimates derived from the LPJ-GUESS photosynthesis model, while \citet{chen2000approaches} estimated photosynthese using the Integrated Terrestrial Ecosystem C-budget model, InTEC). Others measured photosynthesis at the leaf level, through flux towers, or used greenness metrics (NDVI). Growth measurements vary across disciplines and study types, posing a further challenge to an interdisciplinary approach to understanding how growing season length relates to growth. Greenhouse or growth chamber studies and provenance trials were more likely to measure height or biomass, whereas larger scale syntheses and remote-sensed studies are more likely to use metrics of carbon assimilation. 

Aligning across the range and scale of growth metrics will be critical for an integrated understanding of growth-growing season length relationships and implications under continued climate change.  There is decoupling among some metrics of growth. For example, vegetation photosynthesis may be poorly correlated with tree radial growth, and this relationship can vary seasonally \citep{cabon2022cross}. Further, tree radial growth is not a perfect indicator of whole tree growth, since plants allocate carbon to their roots, leaves, reproductive structures, and stores in addition to aboveground biomass. Relationships among different metrics of growth are not simple, so selecting relevant ones and aligning across the most widely used ones will be necessary, though not easy: the relationship  between photosynthesis, radial growth, and carbon uptake has large implications for future carbon sequestration and it remains widely debated \citep{green2022limits}. Further, there is a need to understand how to scale up across these varying metrics- from leaf and individual level to populations, communities, and ecosystems- while incorporating the variation that exists within and across levels.

\section*{Towards standardized measurements \& tests}

 A common framework where researchers aim to report certain relevant drivers (e.g., several temperature and precipitation responses across the growing season) and broaden the metrics they use to encompass a standard set of growth and growing season length metrics....

While our lit review showed that the varying metrics of growth and growing season length can similarly find---or not find---a relationship, we also struggled to make scrape any consistent quantitative estimates---so metrics do matter intensely to helping the field progress. To start the conversation on a standardized set of measurements, we propose:
\begin{enumerate}
\item Ideal measure of growth
\begin{enumerate}
\item Annual increments seem best to me ... no? For long-term forest growth plots, ensuring annual growth measurements paired with phenological measurements would help (most permanent plot data I know of, e.g. FIA and PSP, is collected every 5 years, no phenology). 
\end{enumerate}
\item Ideal measures of wood and vegetative phenology: ideally at the individual level, as satellites cannot differentiate end of season well from herbivory and other factors. Also, given our focus on increment growth, wood phenology is best, but given how hard it is to collect these data we recommend wood phenology people also collect leaf phenology
\begin{enumerate}
\item Xylogenesis folks should measure ....
\item Vegetative: budburst and budset as gold standards for single stages, while photosynthetic measures critical for aligning with satellites etc.. Some of these are time intensive though, so---if you must report more qualitative measures, such as leaf coloring or leaf drop---aim to report several.  
\end{enumerate}
\end{enumerate}

We should also agree on a standardized set of analytical tools. E.g. tree ring analyses --- average away a lot of the interesting things. I’m sure there are issues with phenological data, not sure what they are. ... and people should just report direct tests of the relationship so we don't end up with so many `did not measure' answers (Fig. \ref{fig:heatmaps}). And should we recommend tests for lag effects?

\clearpage
\section*{References}
\bibliography{..//bibtex/grephonbib}
\bibliographystyle{/Users/Lizzie/Documents/git/bibtex/styles/besjournals.bst}

\section*{Figures}


\clearpage
\begin{figure}[h!]
%\includegraphics[width=0.8\textwidth]{..//figures/heatmaps/heatmap_extbymethod.pdf}
%\includegraphics[width=0.8\textwidth]{..//figures/heatmaps/heatmap_endobymethod.pdf}
\includegraphics[width=0.8\textwidth]{..//figures/heatmaps/heatmap_combined_endo&exo.pdf}
\caption{Heatmaps from our literature review showing the prevalence of external (left) and internal drivers considered by method (in development).}
\label{fig:heatmapssupp}
\end{figure}


\clearpage
\begin{figure}[h!]
\includegraphics[width=1\textwidth]{..//figures/speciesnums_finds.pdf}
\caption{Results are generally inconsistent across species (in development).}
\label{fig:sppfinds}
\end{figure}

\clearpage
\begin{figure}[h!]
\includegraphics[width=0.7\textwidth]{..//analyses/growthxelevationetc/figures/grbyelev_rwvsbai_big}
\caption{Growth by elevation relationships for Mount Rainier (in development).}
\label{fig:growelevMORA}
\end{figure}

\end{document}