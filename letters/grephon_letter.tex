\documentclass[11pt,a4paper]{article}
\usepackage[top=1.00in, bottom=1.0in, left=1in, right=1in]{geometry}
\usepackage{graphicx}
\usepackage{sectsty,setspace,natbib,wasysym} 

\begin{document}
\bibliographystyle{/Users/Lizzie/Documents/EndnoteRelated/Bibtex/styles/naturemag}
\begin{figure}[htbp]
\hspace*{14cm}                                                           
\hspace{-35ex} \includegraphics[width=0.5\textwidth]{/Users/Lizzie/Documents/Professional/images/letterhead/ubc/Faculty of forestry.png}
\end{figure}
\vspace{-10ex}
\begin{small}
\noindent 2424 Main Mall \\
\noindent Vancouver, BC Canada V6T 1Z4\\
\noindent Ph: 604.827.5246\\
\end{small}
\vspace{2ex}\\
\pagenumbering{gobble}
\noindent Dear Dr. Wake: % https://www.nature.com/nclimate/editors
\vspace{1.5ex}\\
Please consider our manuscript, ``Why longer seasons with climate change may not increase tree growth,'' for consideration as a Perspective for \emph{Nature Climate Change}. 
\vspace{1.5ex}\\
The idea that longer seasons lead to increased plant growth is an intuitive tenet across multiple fields of biology, and a critical assumption of most global climate models \citep{friedlingstein2022global}. However, a suite of recent studies have challenged this assumption \citep[e.g.][]{dow2022warm,green2022limits}, increasing concerns that future climate change impacts could be underestimated due to methodological factors that influence the relationship between growth and growing season length\citep{green2022limits,korner2023four}.
\vspace{1.5ex}\\
To address this growing debate, we present a Perspective that reviews recent literature to understand the common methods and metrics different fields use to understand the underlying biological mechanisms that may limit how trees grow as climate change extends seasons. We find no evidence for a widespread disconnect between growth and growing season length, and no clear trends by method, or over time. Instead, we find high variation in observed relationships and a pervasive disciplinary split between studies in which biological mechanisms are considered and on which species. We argue that this division, combined with a lack of insights from community and phylogenetic ecology that suggest predictable---and substantial---variation in growth $\times$ season length relationships across species, limits current progress on understanding what underlies different results in different studies and limits progress. But these roadblocks also provide the path to rapid advances. 
\vspace{1.5ex}\\
We outline how increased cross-disciplinary efforts could help the field rapidly develop a unified theoretical framework to predict when, where and how climate change may increase tree growth. After organizing and reviewing the current major drivers controlling tree growth in response to season length, we show how these drivers offer a set of testable hypotheses that could rapidly advance progress. We show how existing data sets and ongoing experiments could provide tests of variation in growth---and potentially controllers on it---across individual to species and ecosystem scales, while new experiments and models can compare effects of external versus internal drivers on growth. 
\vspace{1.5ex}\\
This Perspective benefits from an interdisciplinary authorship team, leveraging expertise from dendrochonology, community ecology, physiology and phylogenetics and we believe could reshape research into tree growth with climate change and we hope that you will find it suitable for publication in \emph{Nature Climate Change}. This manuscript is not under consideration elsewhere, and all authors approved of this version for submission. 
\vspace{1.5ex}\\
Sincerely,\\

\includegraphics[scale=1]{/Users/Lizzie/Documents/Professional/Vitas/Signatures/SignatureLizzieSm.png} \\

\noindent Elizabeth M Wolkovich\\
Associate Professor of Forest \& Conservation Sciences\\ 

\newpage
\bibliography{..//..//bibtex/grephonbib.bib}
\end{document}



% \signature{Elizabeth M Wolkovich}
\address{Forest and Conservation Sciences\\
University of British Columbia\\
2424 Main Mall\\
Vancouver, BC V6T 1Z4}
